\documentclass[ border=2pt]{standalone}
\usepackage{tikz}
\usepackage[utf8]{inputenc}
\usepackage{pgfplots,mathtools}
\usetikzlibrary{arrows,shapes,positioning}
\usepgfplotslibrary{fillbetween}
\usepackage{textcomp}
\usetikzlibrary{colorbrewer}
\begin{document}
	
\usepgfplotslibrary{fillbetween}
\usetikzlibrary{colorbrewer}
\usepgfplotslibrary{colorbrewer}
%Multi-hue
\definecolor{m5}{RGB}{240,249,232}
\definecolor{m4}{RGB}{186,228,188}
\definecolor{m3}{RGB}{123,204,196}
\definecolor{m2}{RGB}{67,162,202}
\definecolor{m1}{RGB}{8,104,172}
\def\layersep{2.5cm}
%Single-hue
\definecolor{s7}{RGB}{239,243,255}
\definecolor{s6}{RGB}{198,219,239}
\definecolor{s5}{RGB}{158,202,225}
\definecolor{s4}{RGB}{107,174,214}
\definecolor{s3}{RGB}{66,146,198}
\definecolor{s2}{RGB}{33,113,181}
\definecolor{s1}{RGB}{8,69,148}


	\begin{tikzpicture} [envel/.style={shape=ellipse, draw, inner sep=-0.2cm}]

\tikzstyle{neuron}=[rectangle,rounded corners,draw=black, fill=s3!50,very thick, inner sep=0.5em, minimum size=2em, text centered]
\tikzstyle{input}=[rectangle,rounded corners,draw=black, fill=s1!50,very thick, inner sep=0.5em, minimum size=2em, text centered]
\tikzstyle{output}=[rectangle,rounded corners,draw=black, fill=s6!50,very thick, inner sep=0.5em, minimum size=2em, text centered]
\tikzstyle{outputstart}=[circle,draw=white,very thick, inner sep=1em, minimum size=3em, text centered,scale=0.5]
\tikzstyle{line} = [ draw, -latex']  
% Draw the input layer nodes

\node[input]at (0.5,1.25) (input1) {};  
\node[input] at (0.5,-0.25)(input2) {};  

%Reservoir

\node[neuron] at (3,1.9) (neuron1) {};  
\node[neuron] at (5,3) (neuron2) {};  
\node[neuron] at (6.4,-2) (neuron3) {};  
\node[neuron] at (3,-0.9) (neuron4) {};  
\node[neuron] at (7.5,1.9) (neuron5) {};  
\node[neuron] at (6,1.9) (neuron6) {}; 
\node[neuron] at (5.5,0) (neuron7) {}; 
\node[neuron] at (8,0) (neuron8) {}; 
% Output
\node[output,right = 9 cm of input1] (output1) {};  
\node[output,right = 9 cm of input2] (output2) {};  
\node[outputstart] at  (8.5,2.4) (outputstart1) {};  
\node[outputstart] at  (8.5,-1.4) (outputstart2) {};  
\node[outputstart] at  (9,0.5) (outputstart) {};  
%Arrow
\node[ellipse, draw, thick,
fit=(neuron1) (neuron2) (neuron3)(neuron4)(neuron5)(neuron6)(neuron7), envel] {};

\draw[->, very thick] (input1.north east)--(1.6,2);
\draw[->, very thick] (input2.north east)--(1.6,0.3);
\draw[->, very thick] (input1.south east)--(1.6,0.7);
\draw[->, very thick] (input2.south east)--(1.6,-1.0);
\draw[dashed, very thick] (input2.north)--(input1.south);

%Hidden
\draw[->, very thick] (neuron2)--(neuron1);
\draw[->, very thick] (neuron4)--(neuron3);
\draw[->, very thick] (neuron1)--(neuron4);
\draw[->, very thick] (neuron1)--(neuron6);
\draw[->, very thick] (neuron7)--(neuron5);
\draw[->, very thick] (neuron3)--(neuron5);
\draw[->, very thick] (neuron3)--(neuron7);
\draw[->, very thick] (neuron5)--(neuron6);
\draw[->, very thick] (neuron6)--(neuron2);
\draw[->, very thick] (neuron5)--(neuron8);
\draw[->, very thick] (neuron3)--(neuron8);
%\draw[->, very thick] (neuron1)--(neuron7);
\draw[->, very thick] (neuron7) to [bend right =10] (neuron1);
\draw[->, very thick] (neuron1) to [bend right = 10] (neuron7);
\draw[->, very thick] (neuron2) to [out=100,in=160,looseness=8] (neuron2);
\draw[->, very thick] (neuron3) to [out=300,in=230,looseness=4] (neuron3);
\draw[->, very thick] (neuron4) to [bend right =10] (neuron7);
\draw[->, very thick] (neuron5) to [bend right =10] (neuron2);
%Output

\draw[->, very thick,dashed] (outputstart1) to [bend right =10] (output1);
\draw[->, very thick] (output1) to [bend right =10] (outputstart1);

\draw[->, very thick,dashed] (outputstart2) to [bend right =10] (output2);
\draw[->, very thick] (output2) to [bend right =10] (outputstart2);

\draw[->, very thick,dashed] (outputstart) to [bend right =10] (output1);
\draw[->, very thick] (output1) to [bend right =10] (outputstart);

\draw[->, very thick,dashed] (outputstart) to [bend right =10] (output2);
\draw[->, very thick] (output2) to [bend right =10] (outputstart);

\draw[->, very thick,dashed] (output1) to [out=30,in=330,looseness=8] (output1);
\draw[->, very thick,dashed] (output2) to [out=30,in=330,looseness=8] (output2);


%Label
\node[label] at (1,5) (neuron7) {K input units}; 
\node[label] at (5,5) (neuron8) {L hidden units}; 
\node[label] at (10.5,5) (neuron7) {M output units}; 
\end{tikzpicture}

	
\end{document}